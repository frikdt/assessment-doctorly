\documentclass[oneside,titlepage,fleqn,a4paper]{article}
\pagestyle{myheadings}
\setlength{\evensidemargin}{0mm} \setlength{\oddsidemargin}{0mm}
\setlength{\textheight}{250mm} \setlength{\textwidth}{150mm}
\setlength{\topmargin}{0mm} \setlength{\headheight}{3mm}
\setlength{\headsep}{0mm} \setlength{\topskip}{0mm}
\setlength{\footskip}{5mm}

\usepackage{amssymb}
\usepackage{amsmath}
%\usepackage{curves}
\usepackage{epsfig}
%\usepackage{graphicx}
\usepackage{array,etoolbox}

\usepackage[bookmarks]{hyperref}
\hypersetup{colorlinks = true, linkcolor = red}

% pass amssymb option to resolve \square redefined conflict
\usepackage[amssymb]{SIunits}

%\preto\tabular{\setcounter{magicrownumbers}{0}}
\newcounter{magicrownumbers}
\newcommand\numberedrow{\stepcounter{magicrownumbers}\arabic{magicrownumbers}}
\newcommand\newrow{\hline \numberedrow}

\begin{document}

\begin{titlepage}
\vspace{100mm}
\begin{center}
DOCTORLY TECHNICAL ASSESSMENT \\
\end{center}
\end{titlepage}

\tableofcontents
\newpage
\section{Task \#3: Monitoring Production Servers}
\subsection{Introduction}
When faced with the task of monitoring production Ubuntu instances, three important considerations come to mind for me:
\begin{enumerate}
\item Choice of tools \& technologies to use.
\item How to implement the chosen technologies as a running system
\item Setting up proper channels to alert the relevant administrators of any possible issues.
\end{enumerate}
Below, I will briefly discuss each of these in more detail.

\subsection{Monitoring Tools \& Technologies}
I have created and configured metrics \& monitoring previously, using a \emph{Prometheus-Grafana} stack (and in one case an additional \emph{blackbox\_exporter} component for monitoring legacy web and database servers), and this has worked extremely well.

\subsection{Implementation}
My implementation of choice for the monitoring stack would be
\begin{itemize}
\item Deploying the monitoring stack in Kubernetes.
\item Using Helm charts for deploying the different components.
\end{itemize}

Alternatively, I'd been considering the creation of a Docker cluster (docker compose) for the monitoring stack if Kubernetes or equivalent was not available to me.

\subsection{Alerts}
The third consideration would be to set up instant alerting with high visibility.

In my experience email alerts, while useful for detail analysis \& troubleshooting, lacks immediacy: People tend to ignore emails or they are simply too occupied with other work to notice an email alert in reasonable time.

I would therefore prefer to set up an instant alert channel using tools such as \emph{MS Teams, Opsgenie, etc.} together with email alerting.

I think it is also very important that the metrics set up in Prometheus \& Grafana, should be relevant, accurate, and as easy as possible for a viewer to understand; hand-in-hand with this, I would take care that alert messages are accurate and easy to understand, rather than vague and/or misleading.

\newpage
\section{Task \#4: A Basic CI/CD Pipeline}
\subsection{Introduction}
In my current role as DevOps Engineer, I have created and updated a \emph{GitLab} CI/CD pipeline for deploying API's and a website, and I would follow the same approach here.

\subsection{Approach \& Assumptions}
\begin{enumerate}
\item The application is an API written in \emph{Typescript/Javascript}; I will call it \textbf{api-fdt}.
\item The \emph{Rush} build tool is used to build, run tests, and publish the application.
\item The application is published a as Docker image that is pushed to the GitLab container registry, from which it may be pulled by a Helm deployment into Kubernetes.
\item The \texttt{package.json} file for the application contains a set of scripts to manage building, testing, publishing, and pushing the resulting Docker image to the GitLab container registry.
\item The CI/CD pipeline will have 3 stages:
\begin{itemize}
\item Build
\item Test
\item Publish
\end{itemize}
\item The pipeline is configured in the \emph{gitlab-ci} yaml file.
\end{enumerate}


\subsection{Reference Material}
In the \texttt{task-4/} directory I have included some relevant files - extracts that still need to be expanded and tested, but could serve as scaffolding and a starting point for the pipeline implementation.
\begin{description}
\item[package.json:] Containing scripts that could be executed by \texttt{rush} or \texttt{rushx} to produce the end-product.
\item[dotgitlab-ci.yml:] Renamed \texttt{.gitlab-ci.yml} for the sake of visibility; this is the GitLab CI/CD pipeline configuration file.
\item[bash helper scripts:] Some example helper scripts (in the \texttt{task-4/helper-scripts/} directory), that are being called by some of the \texttt{package.json} scripts.
\item[readme.md:] A basic document explaining to users how the CI/CD pipeline works.
\end{description}

\newpage
\section{Task \#5: Helm Chart for a WordPress Application}
\subsection{Introduction}
In my current role I have actually done several WordPress deployments into AWS Kubernetes.

For this task I would like to submit an example Helm chart based on one of these deployments.

\subsection{Description}
\begin{enumerate}
\item The chart is based on \emph{WordPress packaged by Bitnami}.
\item Unfortunately, due to me running out of time in this assessment, I haven't been able to do a real deployment and test it.
\item I have, however, included my modified chart and values file in the directory \texttt{task-5/helm/}
\item I also include the output from the commands below in the directory \texttt{task-5/output/}
\end{enumerate}
\begin{itemize}
\item Rendered templates: \\
\begin{scriptsize}
\texttt{helm template -f values.yaml chart/wordpress |\& tee helm-template.txt}
\end{scriptsize}
\item Install dry-run: \\
\begin{scriptsize}
\texttt{helm install --dry-run -f values.yaml doctorly-demo chart/wordpress |\& tee helm-install-dry-run.txt}
\end{scriptsize}
\end{itemize}

% \begin{figure}[p]
%   {
%     \centering
%     \caption{Simulation Manager}
%     \label{f01}
%     \includegraphics[width=150mm, height=89.285mm]{images/eps/001.eps}
%   }
% \end{figure}

\end{document}
